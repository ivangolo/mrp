Cómo fue implementando, interesa la implementación más que el algoritmo
genérico, es decir, si se tiene que implementar SA, lo que se espera
es que se explique en pseudo código la estructura general y en párrafo
explicativo cada parte cómo fue implementada para su caso particular,
si se utilizan operadores se debe explicar por qué se utilizó ese
operador, si fuera el caso de una técnica completa, si se utiliza
recursión o no, etc. \textbf{Recuerde utilizar pseudocódigo para mostrar
cada algoritmo y separar la explicación de su algoritmo en secciones
para que se logre un mejor entendimiento. Además, en este punto no
se espera que se incluya código, eso va aparte.}

\begin{algorithm}
	\caption{My algorithm}\label{euclid}
	\begin{algorithmic}[1]
		\Procedure{MyProcedure}{}
		\State $\textit{stringlen} \gets \text{length of }\textit{string}$
		\State $i \gets \textit{patlen}$
		\BState \emph{top}:
		\If {$i > \textit{stringlen}$} \Return false
		\EndIf
		\State $j \gets \textit{patlen}$
		\BState \emph{loop}:
		\If {$\textit{string}(i) = \textit{path}(j)$}
		\State $j \gets j-1$.
		\State $i \gets i-1$.
		\State \textbf{goto} \emph{loop}.
		\State \textbf{close};
		\EndIf
		\State $i \gets i+\max(\textit{delta}_1(\textit{string}(i)),\textit{delta}_2(j))$.
		\State \textbf{goto} \emph{top}.
		\EndProcedure
	\end{algorithmic}
\end{algorithm}

\begin{algorithm}
	\caption{CH election algorithm}
	\label{CHalgorithm}
	\begin{algorithmic}[1]
		\Procedure{CH\textendash Election}{}
		\For{each node $i$ \Pisymbol{psy}{206} $N$ }
		\State Broadcast HELLO message to its neighbor
		\State let $k$ \Pisymbol{psy}{206} $N1$ ($i$) U {$i$} be s.t
		\State QOS($k$) = max {QOS($j$) \textbar $j$ \Pisymbol{psy}{206} $N1$($i$)  U $i$}
		\State MPRSet($i$) = $k$
		\EndFor
		\EndProcedure
	\end{algorithmic}
\end{algorithm}