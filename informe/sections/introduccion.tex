%Una explicación breve del contenido del informe, es decir, detalla: Propósito, Estructura del Documento, Descripción (muy breve) del Problema y Motivación.

Los centros de datos (\textit{Data centers}), compuestos de servidores y otros equipos, se han convertido en parte sustancial del funcionamiento de las organizaciones modernas, ya sea desde pequeñas o medianas empresas, hasta gigantes corporativos como Amazon, Facebook y Google entre otros, que ofrecen sus servicios basados en el concepto de la ``nube''. Esto viene de la mano con un incremento explosivo del contenido digital, el comercio electrónico y el tráfico en Internet, reflejo de que la carga de trabajo sobre los \textit{Data centers} se ha incrementado. Sin embargo, el funcionamiento de éstos últimos demandan una gran cantidad de electricidad, convirtiéndolos en uno de los consumidores de energía con mayor crecimiento, especialmente en los Estados Unidos. Este escenario se enmarca además, en la mayor preocupación de la sociedad por la sustentabilidad y el cuidado del medio ambiente, en función de los gases de efecto invernadero emitidos a la atmósfera. \\
Uno de los factores que presenta los más grandes problemas y oportunidades para el ahorro de la energía, es el de la sub-utilización de los servidores en los \textit{Data centers}.
Cuando un servidor realiza más trabajo, más eficiente es, tal como un bus que utiliza menos bencina por pasajero cuando lleva cincuenta personas, en comparación a cuando lleva solo unos cuantos. No obstante, el servidor promedio opera a no más del 12 o 18\% de su capacidad. Incluso en estado ocioso, se produce un consumo importante de energía dado al esquema 24/7 de funcionamiento. Para poner esto en perspectiva, gran parte de la energía consumida en los Estados Unidos por los \textit{Data centers} es usada para abastecer a más de 12 millones de servidores, que hacen poco o nada de trabajo durante la mayor parte del tiempo. Aunque los proveedores de la nube a gran escala logran tasas de utilización más altas (entre el 40 a 70\%), incluso estos no son consistentes en alcanzar esas tasas. Nuevas investigaciones desde Google indican que los clusters de servidores típicos promedian en cualquier lugar desde un 10 a un 50\% de uso \cite{DCEfficiencyAssessment}. \\
En ese contexto, se define en el año 2012  el \mrp, desafío propuesto por la French Operational Research and Decision Support Society (ROADEF) y la European Operational Research Society (EURO), en colaboración con Google \cite{2012ProblemDefinition}. El objetivo de este desafío es poder optimizar la utilización de los recursos (unidades de procesamiento CPU, memoria RAM, ancho de banda y almacenamiento en disco entre otros) de un conjunto de máquinas o servidores, que serán asignados para ejecutar un conjunto de procesos. \\

Este documento tiene como objetivo presentar un estudio profundo acerca del \mrp, dividiéndose de la siguiente manera:
\begin{itemize}
	\item En la sección \ref{sec:definicion} se realizará una descripción del problema, tomando en cuenta las variables involucradas, restricciones y objetivos.
	\item En la sección \ref{sec:estado} se presenta el ``estado del arte'' en torno al problema, realizando una inspección de las distintas técnicas desarrolladas hasta la fecha para resolverlo, poniendo hincapié en las que han presentado mejores resultados. 
	\item En la sección \ref{sec:modelo} se presentan dos modelos matemáticos, uno de programación entera mixta y otro de programación con restricciones.
	\item En la sección \ref{sec:conclusiones} se presenta un resumen y algunas consideraciones del presente trabajo, haciendo referencia a las mejores técnicas de resolución del problema, considerando además el trabajo futuro alrededor del \mrp.
\end{itemize}
  

