%Uno o m\'as modelos matem\'aticos para el problema, idealmente indicando el espacio de b\'usqueda para cada uno. Cada modelo debe estar correctamente referenciado, adem\'as no debe ser una imagen extraida. Tambi\'en deben explicarse en detalle cada una de las partes, mostrando claramente la funci\'on a maximizar/minimizar, variables y restricciones. Tanto las f\'ormulas como las explicaciones deben ser consistentes.
Para ambos modelos presentados a continuación, la notación referente a los nombres de los conjuntos de máquinas, procesos, requerimientos, servicios, localizaciones, vecindarios y triplas, será igual a la presentada en la sección \ref{sec:definicion}.
\subsection{Primer Modelo}
El siguiente modelo está basado en  MIP (Mixed Integer Programming) o Programación entera mixta, extraído del trabajo realizado por Mason et al. \cite{masson2013iterated}.

\subsubsection{Variables}
\begin{itemize}
	\item $x_{m,p}$: Toma el valor de uno si el proceso $p$ se ejecuta en la máquina $m$. Cero en caso contrario.
	\item $y_{l,s}$: Toma el valor de uno si al menos un proceso del servicio $s$ se ejecuta en la localización $l$. Cero en caso contrario.
	\item $z_{n,p}$: Toma el valor de uno si el proceso $p$ está en el vecindario $n$.
\end{itemize}

\subsubsection{Parámetros}
\begin{itemize}
	\item $C_{m,r}$: Indica el la capacidad del recurso $r$ en la máquina $m$.
	\item $SC_{m,r}$: Indica el la capacidad segura del recurso $r$ en la máquina $m$.
	\item $R_{p,r}$: Indica la cantidad del recurso $r$ que requiere el proceso $p$.
	\item $SM_{s}$: \textit{Spreadmin}, indica la cantidad mínima de localizaciones distintas en que los procesos del servicio $s$ deben ejecutarse.
	\item $\Phi_{r}$: Es el peso del costo de carga del recurso $r$.
	\item $\Phi_{b}$: Es el peso del costo de carga de balance de la tripla $b$.
	\item $T^b$: Es la proporción entre los recursos pertenecientes a una tripla $b$, para el cálculo del costo de balance.
	\item $\Phi_{p}$: Es el peso del costo de mover un proceso $p$.
	\item $\Phi_{s}$: Es el peso del costo referente al máximo número de procesos movidos según servicio $s$.
	\item $\Phi_{m_1,m_2}$: Es el peso del costo de movimiento de máquina, al mover un proceso $p$ entre las máquinas $m_1$ y $m_2$.
	\item $D^s$: es el conjunto de servicios de los cuales depende el servicio $s$.
\end{itemize}

\subsubsection{Función objetivo}
La función objetivo es la correspondiente a la expresada en la sección \ref{sec:definicion}, pero formalizada según el presente modelo MIP:\\
\begin{align}
	\label{modelo1:loadcost}& \sum_{r \in R}\Phi_{r}\sum_{m \in M}max \{ 0,(\sum_{p \in P}x_{p,m}R_{p,r}) - SC_{m,r} \} \\
	\label{modelo1:balancecost}&+ \sum_{b \in B}\Phi_{b}\sum_{m \in M}max\{0, T^b(C_{m,r_{1}^b} - \sum_{p \in P}x_{p,m}R_{p,r_{1}^b}) - (C_{m,r_{2}^b} - \sum_{p \in P}x_{p,m}R_{p,r_{2}^b})\} \\
	\label{modelo1:processandmachinemovecost}&+ \sum_{p \in P}\sum_{m \in M \backslash M_0(p)}(\Phi_{p} + \Phi_{M_0(p),m})x_{m,p} \\
	\label{modelo1:servicemovecost}&+ \Phi_{s}max(\sum_{p \in s}\sum_{m \in M \backslash M_0(p)}x_{m,p})
\end{align}
\\
\begin{itemize}
	\item \eqref{modelo1:loadcost} es el costo de carga total.
	\item \eqref{modelo1:balancecost} es el costo de balance total.
	\item \eqref{modelo1:processandmachinemovecost} es la suma de los costos de movimiento de proceso y de máquina.
	\item \eqref{modelo1:servicemovecost} es el costo de movimiento de servicio.
\end{itemize}


\subsubsection{Restricciones}
\begin{align}
	\label{modelo1:r1} &\sum_{m \in M}x_{m,p} = 1\;; \quad \forall p \in \mathcal{P} \\ 
	\label{modelo1:r2} &\sum_{p \in P}x_{m,p}R_{p,r} \leq C_{m,r}\;; \qquad \forall m \in \mathcal{M}, r \in R\backslash TR \\
	\label{modelo1:r3} &\sum_{p \in P, M_0(p)\neq m}x_{m,p}R_{p,r}+\sum_{p\in P, M_0(p)=m}R_{p,r} \leq C_{m,r}\;; \qquad \forall r \in TR, m \in \mathcal{M} \\
	\label{modelo1:r4} &\sum_{p \in P} \leq 1\;; \qquad \forall m \in \mathcal{M}, s \in \mathcal{S} \\
	\label{modelo1:r5} &\sum_{p \in s}\sum_{m \in l}x_{m,p} \geq y_{l,s}\;; \qquad \forall l \in \mathcal{L}, s \in \mathcal{S} \\
	\label{modelo1:r6} &\sum_{l \in L}y_{l,s} \geq SM_{s}\;; \qquad \forall s \in \mathcal{S} \\
	\label{modelo1:r7} & z_{n,p} = \sum_{m \in n}x_{m,p}\;; \qquad \forall n \in \mathcal{N}, p \in \mathcal{P} \\
	\label{modelo1:r8} & z_{n,p} \leq \sum_{k \in s_b}z_{n,k}\;; \qquad \forall n \in \mathcal{N}, p \in \mathcal{P}, s_b \in D^{s(p)} \\
	\label{modelo1:r9} & x_{m,p} \in \{0,1\}\;; \qquad \forall m \in \mathcal{M}, p \in \mathcal{P} \\
	\label{modelo1:r10} & x_{l,s} \in \{0,1\}\;; \qquad \forall l \in \mathcal{L}, s \in \mathcal{S} \\
	\label{modelo1:r11} & z_{n,p} \in \{0,1\}\;; \qquad \forall n \in \mathcal{N}, p \in \mathcal{P}
\end{align}
\\
Donde cada restricción corresponde a:
\begin{itemize}
	\item \eqref{modelo1:r1} garantiza que cada proceso sea asignado a una sola máquina.
	\item \eqref{modelo1:r2} restricción de capacidad, establece que el consumo total de cada recurso en cada máquina no debe superar la capacidad de ésta.
	\item \eqref{modelo1:r3} para cada recurso de uso transitorio, su utilización no debe exceder la capacidad dispuesta por las máquinas.
	\item \eqref{modelo1:r4} restricción de conflicto, previene que dos procesos pertenecientes al mismo servicio, sean asignados a la misma máquina.
	\item \eqref{modelo1:r5} y \eqref{modelo1:r6}, relativas al cumplimiento de las restricciones de dispersión. La primera establece que la variable $y_{l,s}$ debe ser cero cuando ningún proceso del servicio $s$ se ejecuta en alguna máquina perteneciente a la localización $l$. La segunda establece que el número de localizaciones utilizadas por un servicio $s$, debe ser mayor o igual al mínimo de dispersión $SM_s$.
	\item \eqref{modelo1:r7} y \eqref{modelo1:r8}, relativas al cumplimiento de las restricciones de dependencia. La primera establece si un proceso $p$, perteneciente a un servicio $s$, se ejecuta en algún vecindario determinado. La segunda establece que, si en la anterior restricción $z_{p,n}$ es uno y siendo el servicio $s_b$ dependencia de $s$, al menos un proceso perteneciente a $s_b$ debe ejecutarse en el vecindario $n$.  
	\item \eqref{modelo1:r9}, \eqref{modelo1:r10} y \eqref{modelo1:r11}, establece el dominio o valores posibles que pueden tomar esas variables, que en este caso son binarias.
\end{itemize}

\subsection{Segundo Modelo}
A continuación se presenta un modelo basado en CP (Constraint Programming) o programación por restricciones, propuesto por Mehta et al. \cite{mehta2012comparing}. 

\subsubsection{Variables}
\begin{itemize}
	\item $x_p$: Variable entera que indica cual máquina es asignada al proceso $p$. $x_p \in [0,|\mathcal{M}|-1]$.
	\item $u_{m,r}$: Variable entera que indica el consumo del recurso $r$ en la máquina $m$. $u_{m,r} \in [0,c_{m,r}]$ ($c_{m,r}$ capacidad del recurso $r$ en la máquina $m$).
	\item $t_{m,r}$: Variable entera que indica la utilización transitoria de una máquina $m$ para un recurso transitorio $r$. $t_{m,r} \in [0,c_{m,r}]$
	\item $um_s$: Indica el conjunto de máquinas que son asignadas a los procesos del servicio $s$.
	\item $yu_{s,l}$: Indica el número de máquinas pertenecientes a la localización $l$, que son asignadas a los procesos pertenecientes al servicio $s$.
	\item $nul_s$: Indica el número de localizaciones usadas por los procesos pertenecientes al servicio $s$.
	\item $zu_{s,n}$: Indica el número de máquinas del vecindario $n$ que son usadas por el servicio $s$.
	\item $zm_{s,n}$: Indica el número de servicios que obligan a $n$ ser el vecindario de $s$.
	\item $nmp_s$: Indica cuantos proceso en el servicio $s$ son asignados a nuevas máquinas.
	\item $mmp$: Indica el máximo número de procesos de los servicios, que son movidos a nuevas máquinas.
\end{itemize}

\subsubsection{Restricciones}
\begin{itemize}
	\item \textit{Restricciones de carga:} El uso de un recurso $r \in \mathcal{R}$ en una máquina $m$ es la suma de los recursos requeridos por aquellos procesos $p$ que son asignados a la máquina $m$:
		\begin{equation}\label{modelo2:r1}
			u_{m,r} = \sum_{p \in \mathcal{P} \wedge x_p = m}r_{p,r}\;; \qquad \forall m \in \mathcal{M}, r \in \mathcal{R}
		\end{equation} 
	
	\item \textit{Restricciones de uso transitorio:} La utilización de un recurso transitorio $r \in T$ ($T$ conjunto de recursos de uso transitorio, $T \in \mathcal{R}$) en una máquina $m$ es la suma de $u_{m,r}$ y los recursos requeridos por aquellos procesos $p$ cuya máquina original $o_p$ es $m$ pero la actual es distinta.
		\begin{equation}\label{modelo2:r2}
			t_{m,r} = u_{m,r} + \sum_{o_p=m \wedge x_p \neq o_p}r_{p,r}\;; \qquad \forall m \in \mathcal{M}, r \in T
		\end{equation}
		
	\item \textit{Restricciones de conflicto:} El conjunto de máquinas usadas por los procesos del servicio $s$ es $um_s := {x_p:p \in s}$. Los procesos pertenecientes al mismo servicio $s$ no pueden ejecutarse en la misma máquina:
		\begin{equation}\label{modelo2:r3}
			|s| = |um_s|\;; \qquad \forall s \in \mathcal{S}
		\end{equation}
	
	\item\textit{Restricciones de dispersión:} El número de máquinas en una localización $l$ usadas por un servicio $s$ es $\forall s \in \mathcal{S}, l \in \mathcal{L},yu_{s,l}:=|\{x_p|p\in s \wedge|L(x_p)=l\}|$ ($L(x_p)$ es la localización de la máquina $x_p$). El número de localizaciones utilizadas por los procesos de un servicio $s$ es $nul_{s}:=|\{l|l\in \mathcal{L}\wedge yu_{s,l}>0|$. El número de localizaciones utilizadas por un servicio $s$ debe ser mayor o igual al mínimo de dispersión de $s$ ($spreadmin_s$):
		\begin{equation}\label{modelo2:r4}
			nul_s \geq spreadmin_s
		\end{equation}
	\item\textit{Restricciones de vecindad:} El número de máquinas usadas por un servicio $s$ en un vecindario $n$ es $zu_{sn}:=|\{x_p|p\in s \wedge N(x_p)=n\}|$ ($N(x_p)$ es el vecindario de la máquina $x_p$). El número de servicios que desean que $n$ sea (obligatoriamente) el vecindario del servicio $s$ es $zm_{sn}:=||\{s'|\langle s',s\rangle \in D \wedge zu_{s'n}=0\}$. El número de vecindarios obligatorios del servicio $s$ que no son usados por este es $nmn_s:=|\{n \in N \wedge zm_{sn} > 0 \wedge zu_{sn} = 0 \}|$. Cada vecindario obligatorio de un servicio $s$ debería ser utilizado por $s$:
	\begin{equation}\label{modelo2:r5}
		nmn_s = 0
	\end{equation}
		 
\end{itemize}

\subsubsection{Función Objetivo}
El costo de una máquina se define como la suma de los costos de carga ponderados para todos los recursos (con su correspondiente peso $w_r$) y la suma de todos los costos de balance ponderados  para todas las triplas de balance (con su correspondiente peso $v_b$):
\begin{equation}\label{modelo2:machinecost}
	cost_m = \sum_{r \in \mathcal{R}}max(0,u_{m,r}-sc_{m,r})\cdot w_r + \sum_{b \in \mathcal{B}}max(0,t_ba_{m,r_{b}^1} - a_{m,r_{b}^2})\cdot v_b
\end{equation}
En \eqref{modelo2:machinecost}, $sc_{m,r}$ es la capacidad segura del recurso $r$ en la máquina $m$, $t_b$ es la proporción entre recursos de la tripla $b$ y $a_{m,r}$ es la cantidad disponible del recurso $r$ en la máquina $m$. \\
El costo de un proceso se define como la suma de los movimientos de proceso ponderados (con su correspondiente peso $w^{pmc}$):
\begin{equation}\label{modelo2:processcost}
	cost_p = min(1,|x_p - o_p|)\cdot pmc_p\cdot w^{pmc} + mmc(o_p,x_p)\cdot w^{mmc}
\end{equation}
En la ecuación \eqref{modelo2:processcost}, $mmc(m_1,m_2)$ es el costo de mover un proceso desde la máquina $m_1$ a la $m_2$ y $pmcp_p$ es el costo de mover el proceso $p$ desde su máquina original a otra distinta.\\
El costo de un servicio se define como la suma ponderada del número de procesos movidos de un servicio:
\begin{equation}\label{modelo2:servicecost}
	cost_s = \sum_{p \in s \wedge x_p \neq o_p}w^{sms}
\end{equation}
La función objetivo a minimizar es el costo total:
\begin{equation}\label{modelo2:overallcost}
	minimizar \qquad cost = \sum_{m \in \mathcal{M}}cost_m + (\sum_{p \in \mathcal{P}}cost_p) + \max\limits_{s \in \mathcal{S}}(cost_s)
\end{equation}

