\documentclass[../informe2.tex]{subfiles}
\begin{document}

%Explicación del problema que se va a estudiar, en qué consiste, cuáles son sus variables , restricciones y objetivo(s) de manera general (en palabras, no una formulación matemática). Debe entenderse claramente el problema y qué busca resolver.
%Explicar si existen problemas relacionados.
%Destacar, si existen, las variantes más conocidas.\\

Se dispone de un conjunto  de máquinas o servidores y un conjunto de procesos. El objetivo del \mrp\ es encontrar una asignación de mínimo costo, de cada proceso a una máquina, que optimice la utilización de los recursos que disponen estas últimas, respetando ciertas restricciones. Una descripción detallada del problema se encuentra en \cite{2012ProblemDefinition}. A continuación, se presenta una síntesis de las variables involucradas, restricciones y costos asociados.

\subsection{Variables}
Se tiene un conjunto $\mathcal{M}$ de máquinas, un conjunto $\mathcal{P}$ de procesos y un conjunto $\mathcal{R}$ de recursos. Una solución al problema es un mapeo que asigna a cada proceso $p \in \mathcal{P}$ una única máquina $m \in \mathcal{M}$. El mapeo $M(p) = m$ indica que el proceso $p$ se ejecuta en la máquina $m$. La asignación inicial de un proceso $p$ se denota como $M_0(p)$.


\subsection{Restricciones}

\begin{itemize}

	\item \textbf{Restricciones de capacidad:} Para el conjunto $\mathcal{R}$ de recursos $r$, cada máquina $m$ puede disponer de una capacidad $C(m,r)$ máxima de cada uno de ellos. Los procesos que se ejecutan en cada máquina, consumen un cierto nivel de esos recursos. La utilización $U(m,r)$ de un recurso $r$ en una máquina $m$ no debe exceder la capacidad de éste.

	\item \textbf{Restricciones de conflicto:} Un servicio $s$ es un conjunto de procesos. Sea $\mathcal{S}$ el conjunto de todos los servicios (disjuntos) que particionan $\mathcal{P}$. Los procesos de un servicio $s \in S$ deben ejecutarse en distintas máquinas.

	\item \textbf{Restricciones de dispersión:} Una localización $l$ es un conjunto de máquinas. Sea $\mathcal{L}$ el conjunto de todas las localizaciones (disjuntas) que particionan $\mathcal{M}$. Para cada servicio, se define un número de dispersión, que indica la cantidad mínima de localizaciones en las que los procesos del servicio deben repartirse para su ejecución.

	\item \textbf{Restricciones de dependencia:} Un vecindario $n$ es un conjunto de máquinas. Sea $\mathcal{N}$ el conjunto de todos los vecindarios (disjuntos) que particionan $\mathcal{M}$. Sea un servicio $s_a$ que depende de otro $s_b$. Entonces cada proceso de $s_a$ debe ejecutarse en el vecindario de algún proceso de $s_b$.

	\item \textbf{Restricciones de uso transitorio:} Cuando un proceso es movido de una máquina $m_a$ a otra $m_b$, hay ciertos recursos que son requeridos en ambas máquinas durante el proceso. Estos son los llamados recursos de uso transitorio. Para cada uno de estos últimos, su utilización no debe exceder la capacidad máxima dispuesta por las máquinas. Es importante notar que no existe dimensión del tiempo en este problema, es decir, todos los movimientos de reasignación se realizan al mismo tiempo.

\end{itemize}

\subsection{Costos}
\begin{itemize}
	% TODO remove equations
	\item \textbf{Costos de carga:} En cada máquina $m$, se establece la capacidad segura $SC(m,r)$ de un recurso $r$. Sea $U(m,r)$ el uso del recurso $r$ que producen los procesos que se ejecutan en la máquina $m$. El costo de carga es el exceso de uso de $r$ por sobre la capacidad segura.

	\item \textbf{Costos de balance:} Se define como el desbalance entre la cantidad disponible de dos recursos distintos en un máquina $m$. Para ello, se establece la tripla $b = <r_a,r_b,t>$, donde $t$ indica la proporción que debe existir entre la cantidad disponible del recurso $r_a$ - $A(m,r_a)$ - y la del recurso $r_b$ - $A(m,r_b)$ -. $\mathcal{B}$ es el conjunto de todas esas triplas.

	\item \textbf{Costos de movimiento de procesos:} Se define como el costo de mover un proceso $p$ desde una máquina inicial cualquiera.

	\item \textbf{Costos de movimiento de servicios:} Se define como el máximo número de procesos movidos según servicio.

	\item \textbf{Costos de movimientos de máquina:} Se define como el costo de mover cualquier proceso $p$, desde una máquina $m_a$ en particular a otra $m_b$ distinta. El movimiento de un proceso dentro de una misma máquina tiene un costo cero.

\end{itemize}

\subsection{Objetivo}
La función objetivo es el costo total $CT$, que se define como la suma de todos los costos anteriormente especificados, con sus respectivos \textit{pesos} o ponderaciones. \\

\subsection{Problemas similares}
Algunos problemas similares al \mrp\ son el Generalized Assignment Problem (GAP), que puede ser representado a través de un conjunto de $n$ trabajos y de $m$ máquinas. El objetivo es encontrar una asignación de mínimo costo, donde para cada trabajo, se debe realizar la asignación de una sola máquina respetando una restricción de recursos para esta última. Asignar un trabajo $j$ a una máquina $m$ implica un costo de $c_{m,j}$, consumiendo una cantidad $a_{m,j}$ de algún recurso, donde en cada máquina, existe una capacidad máxima de este recurso. Este problema a demostrado ser \textit{NP-duro}. \\
Otro problema similar es el Bin Packing Problem (BPP), donde el objetivo es básicamente depositar una serie de objetos dentro de ciertos recipientes, de tal forma que empaque sea factible respecto de ciertas restricciones, como por ejemplo de la capacidad de los depósitos, tratando de optimizar alguna función objetivo, como la utilización de la menor cantidad posible de recipientes para almacenar todos los ítemes disponibles. \\
Para cualquiera de los casos anteriores, el \mrp\ se considera como una instancia más restringida de problema.

\end{document}
