\documentclass[../informe2.tex]{subfiles}
\begin{document}
Se necesita saber cómo experimentaron, cómo definieron parámetros,
cómo se fueron modificando, cuáles problemas se trataron, cuáles fueron
las instancias utilizadas, cuáles fueron las características de las
instancias, cuáles fueron los criterios de término del algoritmo.
\textbf{En el caso de las técnicas incompletas, es importante que
ejecute su programa varias veces con distintas semillas aleatorias,
para obtener valores estadísticos de los resultados}.

% Please add the following required packages to your document preamble:
% \usepackage{booktabs}
\begin{table}[h]
	\small
	\centering
	\begin{tabular}{@{}lrrrrrrrrrr@{}}
		\toprule
		Estadística/Instancia & \multicolumn{1}{c}{a1\_1} & \multicolumn{1}{c}{a1\_2} & \multicolumn{1}{c}{a1\_3} & \multicolumn{1}{c}{a1\_4} & \multicolumn{1}{c}{a1\_5} & \multicolumn{1}{c}{a2\_1} & \multicolumn{1}{c}{a2\_2} & \multicolumn{1}{c}{a2\_3} & \multicolumn{1}{c}{a2\_4} & \multicolumn{1}{c}{a2\_5} \\ \midrule
		Procesos              & 100                       & 1000                      & 1000                      & 1000                      & 1000                      & 1000                      & 1000                      & 1000                      & 1000                      & 1000                      \\
		Máquinas              & 4                         & 100                       & 100                       & 50                        & 12                        & 100                       & 100                       & 100                       & 50                        & 50                        \\
		Recursos              & 2                         & 4                         & 3                         & 3                         & 4                         & 3                         & 12                        & 12                        & 12                        & 12                        \\
		Servicios             & 79                        & 980                       & 216                       & 142                       & 981                       & 1000                      & 170                       & 129                       & 180                       & 153                       \\
		Vecindarios           & 1                         & 2                         & 5                         & 50                        & 2                         & 1                         & 5                         & 5                         & 5                         & 5                         \\
		Localizaciones        & 4                         & 4                         & 25                        & 50                        & 4                         & 1                         & 25                        & 25                        & 25                        & 25                        \\
		Balances              & 1                         & 0                         & 0                         & 1                         & 1                         & 0                         & 0                         & 0                         & 1                         & 0                         \\ \bottomrule
	\end{tabular}
	\caption{\small Set A de instancias.}
	\label{table:set-a}
\end{table}

% Please add the following required packages to your document preamble:
% \usepackage{booktabs}
\begin{table}[h]
	\small
	\centering
	\begin{tabular}{@{}lrrrrrrrrrr@{}}
		\toprule
		Estadística/Instancia & b\_1 & b\_2 & b\_3  & b\_4  & b\_5  & b\_6  & b\_7  & b\_8  & b\_9  & b\_10 \\ \midrule
		Procesos              & 5000 & 5000 & 20000 & 20000 & 40000 & 40000 & 40000 & 50000 & 50000 & 50000 \\
		Máquinas              & 100  & 100  & 100   & 500   & 100   & 200   & 4000  & 100   & 1000  & 5000  \\
		Recursos              & 12   & 12   & 6     & 6     & 6     & 6     & 6     & 3     & 3     & 3     \\
		Servicios             & 2512 & 2462 & 15025 & 1732  & 35082 & 14680 & 15050 & 45030 & 4609  & 4896  \\
		Vecindarios           & 5    & 5    & 5     & 5     & 5     & 5     & 5     & 5     & 5     & 5     \\
		Localizaciones        & 10   & 10   & 10    & 50    & 10    & 50    & 50    & 10    & 100   & 100   \\
		Balances              & 0    & 1    & 0     & 1     & 0     & 1     & 1     & 0     & 1     & 1     \\ \bottomrule
	\end{tabular}
	\caption{\small Set B de instancias.}
	\label{table:set-b}
\end{table}

\end{document}

