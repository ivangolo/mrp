\documentclass[10pt]{article}
\usepackage[utf8]{inputenc}
\usepackage[T1]{fontenc}
\usepackage[spanish]{babel}
\usepackage{amsfonts}
\usepackage{amsmath}
\usepackage{graphicx}
\usepackage{url}
\usepackage[top=3cm,bottom=3cm,left=3.5cm,right=3.5cm,footskip=1.5cm,headheight=1.5cm,headsep=.5cm,textheight=3cm]{geometry}
\usepackage{enumitem}
\usepackage{algorithm}
\usepackage{algpseudocode}
\usepackage[bookmarks]{hyperref}
\usepackage{booktabs}
\usepackage{subfiles}
\usepackage{tikz}
\usepackage[simplified]{pgf-umlcd}
\usetikzlibrary{calc, arrows}
\tikzstyle{arrow} = [thick,->,>=stealth]
\graphicspath{{img/}{../img/}}
\newcommand{\mrp}{\emph{Machine Reassignment Problem}}
\newcommand{\greedy}{\emph{Greedy}}
\newcommand{\hillc}{\emph{Hill Climbing}}
\newcommand{\roadef}{ROADEF 2012}
\providecommand{\tabularnewline}{\\}

\begin{document}

\title{Inteligencia Artificial \\ \begin{Large}Informe Final: \mrp\end{Large}}
\author{Iván E. González López.}
\date{\today}
\maketitle


%--------------------No borrar esta sección--------------------------------%
\section*{Evaluación}
\begin{tabular}{ll}
    Mejoras 1ra Entrega (10 \%):  & \underline{\hspace{2cm}}\tabularnewline
    Código Fuente (10 \%):  & \underline{\hspace{2cm}}\tabularnewline
    Representación (15 \%):  & \underline{\hspace{2cm}} \tabularnewline
    Descripción del algoritmo (20 \%):  & \underline{\hspace{2cm}} \tabularnewline
    Experimentos (10 \%):  & \underline{\hspace{2cm}} \tabularnewline
    Resultados (10 \%):  & \underline{\hspace{2cm}} \tabularnewline
    Conclusiones (20 \%):  & \underline{\hspace{2cm}}\tabularnewline
    Bibliografía (5 \%):  & \underline{\hspace{2cm}}\tabularnewline
     & \tabularnewline
    \textbf{Nota Final (100)}:  & \underline{\hspace{2cm}} \tabularnewline
\end{tabular}
%---------------------------------------------------------------------------%
\vspace{2cm}

\begin{abstract}
\subfile{sections/abstract}
\end{abstract}
\section{Introducción}
\label{sec:introduccion}
\subfile{sections/introduccion}
\section{Definición del Problema}
\label{sec:definicion}
\subfile{sections/definicion_del_problema}
\section{Estado del Arte}
\label{sec:estado}
\subfile{sections/estado_del_arte}
\section{Modelos Matemáticos}
\label{sec:modelo}
\subfile{sections/modelo_matematico}
\section{Representación}
\label{sec:representacion}
\subfile{sections/representacion}
\section{Descripción del algoritmo}
\label{sec:algoritmo}
\subfile{sections/algoritmo}
\section{Experimentos}
\label{sec:experimentos}
\subfile{sections/experimentos}
\section{Resultados}
\label{sec:resultados}
\subfile{sections/resultados}
\section{Conclusiones}
\label{sec:conclusiones}
\subfile{sections/conclusiones}
\section{Bibliografía}
\bibliographystyle{plain}
\bibliography{informe2}
\end{document}
